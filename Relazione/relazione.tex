\documentclass[10pt]{article}
\linespread{1.3}
\usepackage[a4paper, total={6in, 8in}]{geometry}
\usepackage{graphicx,multicol,hyperref,wrapfig,ragged2e} % ricordarsi di rimuovere quelle inutili
\hypersetup{
    colorlinks=true,
    linkcolor=black,
    filecolor=magenta,
    urlcolor=cyan,
    pdftitle={Relazione Marangon Matteo}
}
\justifying

\title{Relazione progetto - Programmazione ad Oggetti}
\author{Marangon Matteo - matricola n$^{\circ}$ 2009094}
\date{A.A. 2024/2025}

\begin{document}

\begin{figure}
    \centering
    \includegraphics[width=0.3\textwidth]{./unipdlogo.png}
\end{figure}
\maketitle

\newpage

\tableofcontents  % valutare se necessaria
\newpage

\section{Introduzione} % circa 200 parole, no immagini !
Vapor è un software che permette di visualizzare la propria libreria digitale di software, videogiochi, DLC e colonne sonore. Si possono aggiungere, modificare, cercare e rimuovere gli elementi della libreria, ciascuno con dei propri attributi caratteristici e una differente visualizzazione grafica. È anche possibile importare una propria libreria, modificarla a piacimento e salvarla per un utilizzo continuo nel tempo.
\\Ho scelto questo argomento per interesse personale nel mondo dei videogiochi e, nello specifico, perché ho sempre prestato attenzione a come vengono sviluppati i launcher (Steam, Epic Games, Ubisoft, EA, ecc.) e quali scelte vengono adottate da ciascun produttore. Questi esempi sono stati fonte di ispirazione per le categorie, gli attributi e alcuni elementi di presentazione estetica, benché il progetto sia estremamente ridotto in confronto.
\section{Descrizione del modello}
\section{Polimorfismo}
\section{Persistenza dei dati}
\section{funzionalità implementate}
\section{Rendiconto ore}

\begin{center}
    \begin{tabular}{| c | c | c |} \hline
    Attività & Ore previste & Ore effettive \\\hline
    Studio e progettazione & 8 & 5 \\
    Studio del framework Qt & 8 & 5 \\
    Sviluppo del codice del modello & 12 & 11 \\
    Sviluppo del codice della GUI & 12 & 12 \\
    Test e debug & 5 & 4,5 \\
    Stesura della relazione & 5 & 0,1 \\\hline
    Totale & 40 & 0 \\\hline
    \end{tabular}
\end{center}

\end{document}